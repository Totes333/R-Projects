% Options for packages loaded elsewhere
\PassOptionsToPackage{unicode}{hyperref}
\PassOptionsToPackage{hyphens}{url}
%
\documentclass[
]{article}
\usepackage{amsmath,amssymb}
\usepackage{lmodern}
\usepackage{ifxetex,ifluatex}
\ifnum 0\ifxetex 1\fi\ifluatex 1\fi=0 % if pdftex
  \usepackage[T1]{fontenc}
  \usepackage[utf8]{inputenc}
  \usepackage{textcomp} % provide euro and other symbols
\else % if luatex or xetex
  \usepackage{unicode-math}
  \defaultfontfeatures{Scale=MatchLowercase}
  \defaultfontfeatures[\rmfamily]{Ligatures=TeX,Scale=1}
\fi
% Use upquote if available, for straight quotes in verbatim environments
\IfFileExists{upquote.sty}{\usepackage{upquote}}{}
\IfFileExists{microtype.sty}{% use microtype if available
  \usepackage[]{microtype}
  \UseMicrotypeSet[protrusion]{basicmath} % disable protrusion for tt fonts
}{}
\makeatletter
\@ifundefined{KOMAClassName}{% if non-KOMA class
  \IfFileExists{parskip.sty}{%
    \usepackage{parskip}
  }{% else
    \setlength{\parindent}{0pt}
    \setlength{\parskip}{6pt plus 2pt minus 1pt}}
}{% if KOMA class
  \KOMAoptions{parskip=half}}
\makeatother
\usepackage{xcolor}
\IfFileExists{xurl.sty}{\usepackage{xurl}}{} % add URL line breaks if available
\IfFileExists{bookmark.sty}{\usepackage{bookmark}}{\usepackage{hyperref}}
\hypersetup{
  pdftitle={Chicago Weather Data Preparation and Analysis},
  pdfauthor={Advaith Cheruvu},
  hidelinks,
  pdfcreator={LaTeX via pandoc}}
\urlstyle{same} % disable monospaced font for URLs
\usepackage[margin=1in]{geometry}
\usepackage{color}
\usepackage{fancyvrb}
\newcommand{\VerbBar}{|}
\newcommand{\VERB}{\Verb[commandchars=\\\{\}]}
\DefineVerbatimEnvironment{Highlighting}{Verbatim}{commandchars=\\\{\}}
% Add ',fontsize=\small' for more characters per line
\usepackage{framed}
\definecolor{shadecolor}{RGB}{248,248,248}
\newenvironment{Shaded}{\begin{snugshade}}{\end{snugshade}}
\newcommand{\AlertTok}[1]{\textcolor[rgb]{0.94,0.16,0.16}{#1}}
\newcommand{\AnnotationTok}[1]{\textcolor[rgb]{0.56,0.35,0.01}{\textbf{\textit{#1}}}}
\newcommand{\AttributeTok}[1]{\textcolor[rgb]{0.77,0.63,0.00}{#1}}
\newcommand{\BaseNTok}[1]{\textcolor[rgb]{0.00,0.00,0.81}{#1}}
\newcommand{\BuiltInTok}[1]{#1}
\newcommand{\CharTok}[1]{\textcolor[rgb]{0.31,0.60,0.02}{#1}}
\newcommand{\CommentTok}[1]{\textcolor[rgb]{0.56,0.35,0.01}{\textit{#1}}}
\newcommand{\CommentVarTok}[1]{\textcolor[rgb]{0.56,0.35,0.01}{\textbf{\textit{#1}}}}
\newcommand{\ConstantTok}[1]{\textcolor[rgb]{0.00,0.00,0.00}{#1}}
\newcommand{\ControlFlowTok}[1]{\textcolor[rgb]{0.13,0.29,0.53}{\textbf{#1}}}
\newcommand{\DataTypeTok}[1]{\textcolor[rgb]{0.13,0.29,0.53}{#1}}
\newcommand{\DecValTok}[1]{\textcolor[rgb]{0.00,0.00,0.81}{#1}}
\newcommand{\DocumentationTok}[1]{\textcolor[rgb]{0.56,0.35,0.01}{\textbf{\textit{#1}}}}
\newcommand{\ErrorTok}[1]{\textcolor[rgb]{0.64,0.00,0.00}{\textbf{#1}}}
\newcommand{\ExtensionTok}[1]{#1}
\newcommand{\FloatTok}[1]{\textcolor[rgb]{0.00,0.00,0.81}{#1}}
\newcommand{\FunctionTok}[1]{\textcolor[rgb]{0.00,0.00,0.00}{#1}}
\newcommand{\ImportTok}[1]{#1}
\newcommand{\InformationTok}[1]{\textcolor[rgb]{0.56,0.35,0.01}{\textbf{\textit{#1}}}}
\newcommand{\KeywordTok}[1]{\textcolor[rgb]{0.13,0.29,0.53}{\textbf{#1}}}
\newcommand{\NormalTok}[1]{#1}
\newcommand{\OperatorTok}[1]{\textcolor[rgb]{0.81,0.36,0.00}{\textbf{#1}}}
\newcommand{\OtherTok}[1]{\textcolor[rgb]{0.56,0.35,0.01}{#1}}
\newcommand{\PreprocessorTok}[1]{\textcolor[rgb]{0.56,0.35,0.01}{\textit{#1}}}
\newcommand{\RegionMarkerTok}[1]{#1}
\newcommand{\SpecialCharTok}[1]{\textcolor[rgb]{0.00,0.00,0.00}{#1}}
\newcommand{\SpecialStringTok}[1]{\textcolor[rgb]{0.31,0.60,0.02}{#1}}
\newcommand{\StringTok}[1]{\textcolor[rgb]{0.31,0.60,0.02}{#1}}
\newcommand{\VariableTok}[1]{\textcolor[rgb]{0.00,0.00,0.00}{#1}}
\newcommand{\VerbatimStringTok}[1]{\textcolor[rgb]{0.31,0.60,0.02}{#1}}
\newcommand{\WarningTok}[1]{\textcolor[rgb]{0.56,0.35,0.01}{\textbf{\textit{#1}}}}
\usepackage{graphicx}
\makeatletter
\def\maxwidth{\ifdim\Gin@nat@width>\linewidth\linewidth\else\Gin@nat@width\fi}
\def\maxheight{\ifdim\Gin@nat@height>\textheight\textheight\else\Gin@nat@height\fi}
\makeatother
% Scale images if necessary, so that they will not overflow the page
% margins by default, and it is still possible to overwrite the defaults
% using explicit options in \includegraphics[width, height, ...]{}
\setkeys{Gin}{width=\maxwidth,height=\maxheight,keepaspectratio}
% Set default figure placement to htbp
\makeatletter
\def\fps@figure{htbp}
\makeatother
\setlength{\emergencystretch}{3em} % prevent overfull lines
\providecommand{\tightlist}{%
  \setlength{\itemsep}{0pt}\setlength{\parskip}{0pt}}
\setcounter{secnumdepth}{-\maxdimen} % remove section numbering
\ifluatex
  \usepackage{selnolig}  % disable illegal ligatures
\fi

\title{Chicago Weather Data Preparation and Analysis}
\author{Advaith Cheruvu}
\date{8/31/2021}

\begin{document}
\maketitle

library(rmarkdown) render(``W2L2Cheruvu.Rmd'', output\_format =
``pdf\_document'') \#\# Abstract

This report analyzes the weather data for Chicago from January 1987 to
December 2005. The measured weather parameters were daily temperature
(degrees Celsius), dewpoint (degrees Celsius), amount of PM2.5, amount
of PM10, amount of ozone, and amount of nitrogen trioxide. Data munging,
cleaning and normalization were done in order to prepare the dataset for
further analysis. Histograms of numerical data were made.

\hypertarget{clean-up-and-set-up}{%
\subsection{Clean up and Set up}\label{clean-up-and-set-up}}

Before importing the data, the workspace must be cleared and the working
directory must be set up.

\begin{Shaded}
\begin{Highlighting}[]
\CommentTok{\# Clearing the workspace and setting the working directory}
\FunctionTok{rm}\NormalTok{(}\AttributeTok{list=}\FunctionTok{ls}\NormalTok{())}
\FunctionTok{setwd}\NormalTok{(}\StringTok{"/Users/advai/Documents/DSFS"}\NormalTok{)}
\end{Highlighting}
\end{Shaded}

\hypertarget{obtaining-functions-and-installing-packages}{%
\subsection{Obtaining Functions and Installing
Packages}\label{obtaining-functions-and-installing-packages}}

``myfunctions.R'' is necessary for some of the later code to run.
``tidyverse'' has useful tools that help with plotting and transforming
data.

\begin{verbatim}
# install and load libraries
source("myfunctions.R")
install.packages("tidyverse")
library(tidyverse)
\end{verbatim}

\hypertarget{loading-and-overviewing-the-weather-data}{%
\subsection{Loading and Overviewing the Weather
Data}\label{loading-and-overviewing-the-weather-data}}

Before cleaning the data, we must know what the dataset includes. The
``names'' command shows the names of the columns. The ``summary''
command gives a short summary of each column. The ``str'' command shows
the type of data in each column. The ``dim'' command shows the number of
columns and rows in the dataset. The ``class'' command shows that
weather is a data frame.

\begin{Shaded}
\begin{Highlighting}[]
\CommentTok{\# load weather data}
\NormalTok{weather }\OtherTok{\textless{}{-}} \FunctionTok{read.csv}\NormalTok{(}\AttributeTok{file =} \StringTok{"C:}\SpecialCharTok{\textbackslash{}\textbackslash{}}\StringTok{Users}\SpecialCharTok{\textbackslash{}\textbackslash{}}\StringTok{advai}\SpecialCharTok{\textbackslash{}\textbackslash{}}\StringTok{Documents}\SpecialCharTok{\textbackslash{}\textbackslash{}}\StringTok{DSFS}\SpecialCharTok{\textbackslash{}\textbackslash{}}\StringTok{chicago.csv"}\NormalTok{,}\AttributeTok{header=}\NormalTok{T)}
\CommentTok{\# overview of the data set}
\FunctionTok{names}\NormalTok{(weather)}
\end{Highlighting}
\end{Shaded}

\begin{verbatim}
## [1] "indx"       "city"       "tmpd"       "dptp"       "date"      
## [6] "pm25tmean2" "pm10tmean2" "o3tmean2"   "no2tmean2"
\end{verbatim}

\begin{Shaded}
\begin{Highlighting}[]
\FunctionTok{summary}\NormalTok{(weather)}
\end{Highlighting}
\end{Shaded}

\begin{verbatim}
##       indx          city                tmpd             dptp       
##  Min.   :   1   Length:6940        Min.   :-16.00   Min.   :-25.62  
##  1st Qu.:1736   Class :character   1st Qu.: 35.00   1st Qu.: 27.00  
##  Median :3470   Mode  :character   Median : 51.00   Median : 39.88  
##  Mean   :3470                      Mean   : 50.31   Mean   : 40.34  
##  3rd Qu.:5205                      3rd Qu.: 67.00   3rd Qu.: 55.75  
##  Max.   :6940                      Max.   : 92.00   Max.   : 78.25  
##                                    NA's   :1        NA's   :2       
##      date             pm25tmean2      pm10tmean2        o3tmean2      
##  Length:6940        Min.   : 1.70   Min.   :  2.00   Min.   : 0.1528  
##  Class :character   1st Qu.: 9.70   1st Qu.: 21.50   1st Qu.:10.0729  
##  Mode  :character   Median :14.66   Median : 30.28   Median :18.5218  
##                     Mean   :16.23   Mean   : 33.90   Mean   :19.4355  
##                     3rd Qu.:20.60   3rd Qu.: 42.00   3rd Qu.:27.0010  
##                     Max.   :61.50   Max.   :365.00   Max.   :66.5875  
##                     NA's   :4447    NA's   :242                       
##    no2tmean2     
##  Min.   : 6.158  
##  1st Qu.:19.654  
##  Median :24.556  
##  Mean   :25.232  
##  3rd Qu.:30.139  
##  Max.   :62.480  
## 
\end{verbatim}

\begin{Shaded}
\begin{Highlighting}[]
\FunctionTok{str}\NormalTok{(weather)}
\end{Highlighting}
\end{Shaded}

\begin{verbatim}
## 'data.frame':    6940 obs. of  9 variables:
##  $ indx      : int  1 2 3 4 5 6 7 8 9 10 ...
##  $ city      : chr  "chic" "chic" "chic" "chic" ...
##  $ tmpd      : num  31.5 33 33 29 32 40 34.5 29 26.5 32.5 ...
##  $ dptp      : num  31.5 29.9 27.4 28.6 28.9 ...
##  $ date      : chr  "1/1/87" "1/2/87" "1/3/87" "1/4/87" ...
##  $ pm25tmean2: num  NA NA NA NA NA NA NA NA NA NA ...
##  $ pm10tmean2: num  34 NA 34.2 47 NA ...
##  $ o3tmean2  : num  4.25 3.3 3.33 4.38 4.75 ...
##  $ no2tmean2 : num  20 23.2 23.8 30.4 30.3 ...
\end{verbatim}

\begin{Shaded}
\begin{Highlighting}[]
\FunctionTok{dim}\NormalTok{(weather)}
\end{Highlighting}
\end{Shaded}

\begin{verbatim}
## [1] 6940    9
\end{verbatim}

\begin{Shaded}
\begin{Highlighting}[]
\FunctionTok{class}\NormalTok{(weather)}
\end{Highlighting}
\end{Shaded}

\begin{verbatim}
## [1] "data.frame"
\end{verbatim}

\hypertarget{removing-columns}{%
\subsection{Removing Columns}\label{removing-columns}}

The index and city data in this dataset is unecessary so we must remove
those columns.

\begin{verbatim}
# removing index and city columns
weather <- select(weather, -1:-2)
names(weather)
\end{verbatim}

\includegraphics{C://Users//advai//Documents//DSFS//removingcolumns.png}
\#\# Renaming Columns Some of the columns are ambigous or hard to read.
Using these commands, we can rename them to something more readeable.

\begin{verbatim}
# renaming columns
colnames(weather)<-c("Temp", "Dewpoint","Date","PM25","PM10","O3","NO3")
names(weather)
\end{verbatim}

\includegraphics{C://Users//advai//Documents//DSFS//renamedColumns.png}

\hypertarget{changing-the-date-format}{%
\subsection{Changing the Date Format}\label{changing-the-date-format}}

The date format is a character instead of a number. using these
commands, we can change the date to a number, as well as make it more
readable.

\begin{verbatim}
# Changing the date format
weather$Date <- as.Date(weather$Date, format= "%m/%d/%y")
\end{verbatim}

\hypertarget{normalizing-the-pm-data}{%
\subsection{Normalizing the PM data}\label{normalizing-the-pm-data}}

The PM data is missing some values and it should be normalized (on a
scale of 0-1) to prepare it for data analysis. This applies to both
PM2.5 and PM10.

\begin{verbatim}
# Normalizing PM25 Data
weather$PM25 <- ifelse(is.na(weather$PM25), round(mean(weather$PM25, na.rm=TRUE),3), weather$PM25)
PM25.normalize <- weather$PM25/max(weather$PM25)
weather$PM25 <- PM25.normalize
# Normalizing PM10 Data
weather$PM10 <- ifelse(is.na(weather$PM10), round(mean(weather$PM10, na.rm=TRUE),3), weather$PM10)
PM10.normalize <- weather$PM10/max(weather$PM10)
weather$PM10 <- PM10.normalize
\end{verbatim}

\hypertarget{taking-out-numeric-data}{%
\subsection{Taking out numeric data}\label{taking-out-numeric-data}}

To prepare the data for analysis, we must provide the columns with
numeric data. These commands seperate the numerica data (all the columns
except for the date).

\begin{verbatim}
# Naming and removing numeric data
numerics <- weather[,c(1:2,4:7)]
\end{verbatim}

\hypertarget{observing-correlation}{%
\subsection{Observing Correlation}\label{observing-correlation}}

This is pairwise correlation which helps to find out which data pairs
have recognizeable patterns and are worth analyzing further.

\begin{verbatim}
# Looking at Correlation
pairs(numerics,upper.panel = panel.cor,diag.panel = panel.hist)
\end{verbatim}

\includegraphics{C://Users//advai//Documents//DSFS//pairs.png}

\hypertarget{histograms}{%
\subsection{Histograms}\label{histograms}}

As part of Exploratory Data Analysis (EDA), transformations must be made
to the data to normalize it and compare it to other data. This
transformation is the rank z transformation which results in a normal
distribution. This can be seen for all graphs except for PM25 where the
correlation is not as strong.

\begin{verbatim}
# Transformations and histogram plots
Temp.rz <- rz.transform(weather$Temp)
hist(Temp.rz)
Dewpoint.rz <- rz.transform(weather$Dewpoint)
hist(Dewpoint.rz)
PM25.rz <- rz.transform(weather$PM25)
hist(PM25.rz)
PM10.rz <- rz.transform(weather$PM10)
hist(PM10.rz)
O3.rz <- rz.transform(weather$O3)
hist(O3.rz)
NO3.rz <- rz.transform(weather$NO3)
hist(NO3.rz)
\end{verbatim}

\includegraphics{C://Users//advai//Documents//DSFS//temprz.png}
\includegraphics{C://Users//advai//Documents//DSFS//dewpointrz.png}
\includegraphics{C://Users//advai//Documents//DSFS//pm25rz.png}
\includegraphics{C://Users//advai//Documents//DSFS//pm10rz.png}
\includegraphics{C://Users//advai//Documents//DSFS//o3rz.png}
\includegraphics{C://Users//advai//Documents//DSFS//no3rz.png}

\end{document}
